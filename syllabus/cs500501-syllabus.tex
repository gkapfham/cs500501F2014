%!TEX root=cs500501-syllabus.tex
% mainfile: cs500501-syllabus.tex 
% CS 580 style
% Typical usage (all UPPERCASE items are optional):
%       \input 580pre
%       \begin{document}
%       \MYTITLE{Title of document, e.g., Lab 1\\Due ...}
%       \MYHEADERS{short title}{other running head, e.g., due date}
%       \PURPOSE{Description of purpose}
%       \SUMMARY{Very short overview of assignment}
%       \DETAILS{Detailed description}
%         \SUBHEAD{if needed} ...
%         \SUBHEAD{if needed} ...
%          ...
%       \HANDIN{What to hand in and how}
%       \begin{checklist}
%       \item ...
%       \end{checklist}
% There is no need to include a "\documentstyle."
% However, there should be an "\end{document}."
%
%===========================================================
\documentclass[11pt,twoside,titlepage]{article}
%%NEED TO ADD epsf!!
\usepackage{threeparttop}
\usepackage{graphicx}
\usepackage{latexsym}
\usepackage{color}
\usepackage{listings}
\usepackage{fancyvrb}
%\usepackage{pgf,pgfarrows,pgfnodes,pgfautomata,pgfheaps,pgfshade}
\usepackage{tikz}
\usepackage[normalem]{ulem}
\tikzset{
    %Define standard arrow tip
%    >=stealth',
    %Define style for boxes
    oval/.style={
           rectangle,
           rounded corners,
           draw=black, very thick,
           text width=6.5em,
           minimum height=2em,
           text centered},
    % Define arrow style
    arr/.style={
           ->,
           thick,
           shorten <=2pt,
           shorten >=2pt,}
}
\usepackage[noend]{algorithmic}
\usepackage[noend]{algorithm}
\newcommand{\bfor}{{\bf for\ }}
\newcommand{\bthen}{{\bf then\ }}
\newcommand{\bwhile}{{\bf while\ }}
\newcommand{\btrue}{{\bf true\ }}
\newcommand{\bfalse}{{\bf false\ }}
\newcommand{\bto}{{\bf to\ }}
\newcommand{\bdo}{{\bf do\ }}
\newcommand{\bif}{{\bf if\ }}
\newcommand{\belse}{{\bf else\ }}
\newcommand{\band}{{\bf and\ }}
\newcommand{\breturn}{{\bf return\ }}
\newcommand{\mod}{{\rm mod}}
\renewcommand{\algorithmiccomment}[1]{$\rhd$ #1}
\newenvironment{checklist}{\par\noindent\hspace{-.25in}{\bf Checklist:}\renewcommand{\labelitemi}{$\Box$}%
\begin{itemize}}{\end{itemize}}
\pagestyle{threepartheadings}
\usepackage{url}
\usepackage{wrapfig}
% removing the standard hyperref to avoid the horrible boxes
%\usepackage{hyperref}
\usepackage[hidelinks]{hyperref}
% added in the dtklogos for the bibtex formatting
\usepackage{dtklogos}
%=========================
% One-inch margins everywhere
%=========================
\setlength{\topmargin}{0in}
\setlength{\textheight}{8.5in}
\setlength{\oddsidemargin}{0in}
\setlength{\evensidemargin}{0in}
\setlength{\textwidth}{6.5in}
%===============================
%===============================
% Macro for document title:
%===============================
\newcommand{\MYTITLE}[1]%
   {\begin{center}
     \begin{center}
     \bf
     CMPSC 500 and 501\\Internship Seminar\\
     Fall 2014
     \medskip
     \end{center}
     \bf
     #1
     \end{center}
}
%================================
% Macro for headings:
%================================
\newcommand{\MYHEADERS}[2]%
   {\lhead{#1}
    \rhead{#2}
    %\immediate\write16{}
    %\immediate\write16{DATE OF HANDOUT?}
    %\read16 to \dateofhandout
    \def \dateofhandout {}
    \lfoot{}
    %\immediate\write16{}
    %\immediate\write16{HANDOUT NUMBER?}
    %\read16 to\handoutnum
    \def \handoutnum {1}
    \rfoot{Handout \handoutnum}
   }

%================================
% Macro for bold italic:
%================================
\newcommand{\bit}[1]{{\textit{\textbf{#1}}}}

%=========================
% Non-zero paragraph skips.
%=========================
\setlength{\parskip}{1ex}

%=========================
% Create various environments:
%=========================
\newcommand{\PURPOSE}{\par\noindent\hspace{-.25in}{\bf Purpose:\ }}
\newcommand{\SUMMARY}{\par\noindent\hspace{-.25in}{\bf Summary:\ }}
\newcommand{\DETAILS}{\par\noindent\hspace{-.25in}{\bf Details:\ }}
\newcommand{\HANDIN}{\par\noindent\hspace{-.25in}{\bf Hand in:\ }}
\newcommand{\SUBHEAD}[1]{\bigskip\par\noindent\hspace{-.1in}{\sc #1}\\}
%\newenvironment{CHECKLIST}{\begin{itemize}}{\end{itemize}}


\usepackage[compact]{titlesec}

\begin{document}
\MYTITLE{Syllabus}
\MYHEADERS{Syllabus}{}

\subsection*{Course Instructor}
Dr.\ Gregory M.\ Kapfhammer\\
\noindent Office Location: Alden Hall 108 \\
\noindent Office Phone: +1 814-332-2880 \\
\noindent Email: \url{gkapfham@allegheny.edu} \\
\noindent Twitter: \url{@GregKapfhammer} \\
\noindent Web Site: \url{http://www.cs.allegheny.edu/sites/gkapfham/} 

\subsection*{Instructor's Office Hours}

\begin{itemize}
	\itemsep 0em
	\item Monday: 4:30 pm -- 5:30 pm (15 minute time slots)
	\item Wednesday: 1:00 pm -- 4:00 pm (15 minute time slots)
	\item Thursday: 1:00 pm -- 3:00 pm (15 minute time slots)
	\item Friday: 9:00 -- 10:00 am (10 minute time slots) {\em and} \\ \hspace*{.49in} 1:00 pm -- 2:00 pm (5 minute time slots)
\end{itemize}

\noindent
To schedule a meeting with me during my office hours, please visit my Web site and click the ``Schedule'' link in the
top right-hand corner. Now, you can browse my office hours or schedule an appointment by clicking the correct link and
then reserving an open time slot. 

\subsection*{Course Meeting Schedule}

Neither CMPSC 500 nor 501 have a set meeting time.  Instead, students are required to meet with the course instructor
during office hours for a total of 60 minutes.  These meetings must be scheduled through the course instructor's reservation
system and documented on a meeting record that you submit on the last day of classes (see the ``Grading Policy''
section for more details).

\subsection*{Course Catalogue Description}

\begin{quote}
A corequisite seminar for all students participating in the Internship Program.  Internship students enroll twice, once
prior to and once following the Internship. Computer Science 500 focuses on expectations and planning, leading to the
Internship Proposal required for all students planning an internship.  Computer Science 501 provides the opportunity for
students who have completed the Internship to prepare written and oral reports on the Internship experience. The
student, in consultation with the faculty of the Department of Computer Science, is responsible for arranging for an
acceptable internship.  The courses meet together weekly for one-half a semester. Credit: One semester hour for each
course. Prerequisites: Completion of at least two core courses.
\end{quote}

\subsection*{Course Objectives}

The students in CMPSC 500 will develop a list of internship and career resources, explore their internship and career
interests, and write a proposal for the internship that they will undertake. The course aims to help students
learn more about the field of computing and the positions that are available to computer scientists who graduate from
Allegheny College.  Overall, the course will provide a framework that enables a student to secure an interesting and
meaningful internship.

\noindent
The students in CMPSC 501 will create a list of career resources, explore their career interests, write a final report
reflecting on their internship, and give a public presentation about their internship.  This course gives
students who recently completed an internship a forum for sharing their experiences with both computer science faculty
and students and the college community as a whole.

\subsection*{Performance Objectives}

At the completion of either CMPSC 500 or CMPSC 501, a student should be familiar with resources related to various facets of
internships and careers. Students should have skills that enable them to effectively investigate internships and
careers.  Also, the student should have a better understanding of the computer science sub-disciplines that they may
want to pursue after graduating from Allegheny College. Finally, students should also have improved their writing and
speaking skills.

\subsection*{Required Textbooks}

\noindent
There are no required textbooks for this class.  However, students who want to improve their general and technical writing skills
may consult the following books.

\noindent{\em BUGS in Writing: A Guide to Debugging Your Prose}. Lyn Dupr\'e. Second Edition,  ISBN-10: 020137921X,
ISBN-13: 978-0201379211, 704 pages, 1998.

\noindent{\em Writing for Computer Science}.  Justin Zobel. Second Edition,  ISBN-10: 1852338024, ISBN-13:
978-1852338022, 270 pages, 2004.

\subsection*{Class Policies}

\subsubsection*{Grading}

% Final Resume
% Two Page Career Interest Statement
% Two Page Description of Career Resources
% Secured Internship
% Two Page Internship Proposal
% 
% Students will be expected to "workshop" their resume with the Instructor and the staff in the Office of Career Services
% (OCS). Members of the Office of Career Services will be able to provide the student will invaluable advice concerning
% resumes, internship opportunities, and careers. A student's career interest statement must describe the types of careers
% in which a student is interested. However, this document should go beyond the basic statement of a job description and
% also include information about preferred geographical location, technologies that the student is interested in using,
% types of software the student would like to develop, etc. Each student will be responsible for the creation of a
% document that describes the career and internship resources that they found to be the most valuable. For example, a
% student might create a listing of different Web sites and books that they found useful. However, this listing should
% also include a description of the strengths and weaknesses of these resources. Finally, each student must secure an
% internship and write a short proposal (in conjunction with the staff members at their chosen company) to describe the
% work that they will conduct.
% 

The grade that a student receives in this class will be based on the following categories. All percentages are
approximate and, if the need to do so presents itself, it is possible for the assigned percentages to change during the
academic semester. \\

Grade percentages for CMPSC 500:
\begin{center}
\begin{tabular}{ll}
Class Participation and Instructor Meetings & 20\% \\
Career Interest Statement & 20\% \\
Description of Career Resources & 20\% \\
Internship Proposal & 30\% \\
Secured Internship & 10\% \\
\end{tabular}
\end{center}

\newpage
Grade percentages for CMPSC 501:

\vspace{-.1in}
\begin{center}
\begin{tabular}{ll}
Class Participation and Instructor Meetings & 20\% \\
Internship Retrospective & 25\% \\
Description of Career Resources & 25\% \\
Internship Presentation & 30\% \\
\end{tabular}
\end{center}

\noindent
The aforementioned grading categories have the following definitions:

\begin{itemize}

	\item {\em Class Participation and Instructor Meetings}: All students are required to attend the public
	  presentations given as part of the internship seminar.  Furthermore, all students must meet with the course
	  instructor during office hours for a total of sixty minutes during the Spring 2014 semester.  These meetings
	  must be scheduled through the instructor's reservation system and documented on a meeting record that you
	  submit on the last day of classes. Attendance at the seminar's presentations corresponds to 5\% of the
	  student's grade in this category and the instructor meetings comprise the other 15\%. If $t=60$, then a
	  student who attends $m$ minutes of meetings will receive a grade of $\frac{\max{(t,60)}}{t}\times100$ for the
	  15\% of this grade.
		
	\item {\em Career Interest Statement}: This document must describe the types of careers in which a student has
		interest. This statement should go beyond furnishing a job title and description and also give
		information concerning issues such as preferred geographical location, technologies that the student wants to
		use, and types of software that the student would like to develop. 	

	\item {\em Description of Career Resources}:  Students must prepare a document that describes the career and
		internship resources that they found to be the most valuable during their participation in the internship
		seminar. For instance, students might create a listing of different Web sites and books that they found useful. 
		Beyond simply listing the resources, this document should highlight their strengths, weaknesses, and interconnections. 		

	\item {\em Internship Proposal}: Students must write a short proposal, in conjunction with the staff members of the
		company that will host their internship, to describe the work that they will conduct during the internship.
		This document should furnish as much detail as is possible.  For instance, the proposal should state who will be
		the student's mentor, how the student will integrate into existing team(s) within the company, and the
		student's assigned project(s). 

	\item {\em Secured Internship}: Each student should work hard to secure an internship by the end of the seminar.  If
		students anticipate that they will not be able to secure an internship, they should discuss this matter with the
		course instructor well in advance of the end of the semester.  In some situations, the course instructor may be 
		able to make alternative arrangements for a student who has not yet been able to secure an internship.

	\item {\em Internship Presentation}: After completing their internship, students should give a public presentation
		hosted by the Department of Computer Science. In consultation with the course instructor and the president of
		Allegheny College's student chapter of the ACM, students must schedule a seminar in which they will
		present.  Well in advance of their presentation, students must deliver to the course instructor a copy of their 
		presentation slides.  Students should speak for at least ten minutes and be prepared to answer questions for
		several minutes.     

	\item {\em Internship Retrospective}: Students who have completed their internship should write a short document
		describing the work that they completed.  This document should also reflect on whether the student would suggest
		that other Allegheny College students should also intern at the company that offered their internship;  if the
		company is suitable then this document should furnish contact information for people at the company.  In
		contrast to the internship presentation, this document affords students an opportunity to privately reflect on
		their experiences in a way that might not be appropriate in a public forum.

\end{itemize}

\subsubsection*{Email}

Using your Allegheny College email address, I will sometimes send out class announcements with assignment clarifications
or presentations announcements. It is your responsibility to check your email at least once a day and to ensure that you
can reliably send and receive emails. This class policy is based on the following statement in {\em The Compass}, the
college's student handbook.

\vspace*{-.1in}
\begin{quote}
``The use of email is a primary method of communication on campus. \ldots
All students are provided with a campus email account and address while
enrolled at Allegheny and are expected to check the account on a regular
basis.'' 
\end{quote}
\vspace*{-.15in}

\subsubsection*{Disability Services}

The Americans with Disabilities Act (ADA) is a federal anti-discrimination statute that provides comprehensive civil
rights protection for persons with disabilities.  Among other things, this legislation requires all students with
disabilities be guaranteed a learning environment that provides for reasonable accommodation of their disabilities.
Students with disabilities who believe they may need accommodations in this class are encouraged to contact Disability
Services at 332-2898.  Disability Services is part of the Learning Commons and is located in Pelletier Library.
Please do this as soon as possible to ensure that approved accommodations are implemented in a timely fashion.

\subsubsection*{Honor Code}

The Academic Honor Program that governs the entire academic program at Allegheny College is described in the Allegheny
Course Catalogue.  The Honor Program applies to all work that is submitted for academic credit or to meet non-credit
requirements for graduation at Allegheny College.  This includes all work assigned for this class (e.g., career interest
statement and description of career resources).  All students who have enrolled in the College
will work under the Honor Program.  Each student who has matriculated at the College has acknowledged the following
pledge:

\begin{quote}
I hereby recognize and pledge to fulfill my responsibilities, as defined in the Honor Code, and to maintain the
integrity of both myself and the College community as a whole.  
\end{quote}

\subsection*{Welcome to an Adventure in Computer Science Internships}

A recent article in the {\em Seattle Times} opined that ``computer science is where the future jobs are''.  In this
seminar, you will have the opportunity to search for and share about internship opportunities.  Since computer software
and hardware are pervasive and influential aspects of our society, an internship in computer science effectively
prepares you for a job after graduation.  As we start this class, I invite you to pursue with enthusiasm and
vigor this adventure in internships.

\end{document}
